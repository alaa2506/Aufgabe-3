\documentclass[a4paper,12pt]{article}

\usepackage[utf8]{inputenc} 
\usepackage[T1]{fontenc} 
\usepackage{graphicx}
\usepackage[german]{babel}
\usepackage{geometry}
\geometry{a4paper, margin=0.5in}


\renewcommand{\contentsname}{Inhaltsverzeichnis}

\title{
  {\Large Werkzeuge für das wissenschaftliche Arbeiten:} \\
  {\Large Python for Machine Learning and Data Science}
}
\date{Abgabe: 15.12.2023}

\begin{document}

\maketitle
\thispagestyle{empty}
\hrule
\tableofcontents
\vspace{1cm}
\hrule

\section{Projektaufgabe}
In dieser Aufgabe beschäftigen wir uns mit Objektorientierung in Python. Der Fokus liegt auf der Implementierung einer Klasse, dabei nutzen wir insbesondere auch Magic Methods.

\begin{center}
    \includegraphics[width=0.6\textwidth]{./../diagram/classes_files.svg}\\
    \small Abbildung 1: Darstellung der Klassenbeziehungen
\end{center}

\subsection{Einleitung}
Ein Datensatz besteht aus mehreren Daten, ein einzelnes Datum wird durch ein Objekt der Klasse "DataSetItem" repräsentiert. Jedes Datum hat einen Name (Zeichenkette), eine ID (Zahl) und beliebigen Inhalt. Nun sollen mehrere Daten, Objekte vom Typ "DataSetItem", in einem Datensatz zusammengefasst werden. Sie haben sich schon auf eine Schnittstelle und die benötigen Operationen, die ein Datensatz unterstützen muss, geeinigt. Es gibt eine Klasse "DataSetInterface", die die Schnittstelle definiert und Operationen jedes Datensatzes angibt. Bisher fehlt aber noch die Implementierung eines Datensatzes mit allen Operationen.
Implementieren Sie eine Klasse "DataSet" als eine Unterklasse von "DataSet-Interface".

\subsection{Aufbau}
Es gibt drei Dateien: "dataset.py", "main.py" und "implementation.py". In "dataset.py" befinden sich die Klassen "DataSetInterface" und "DataSetItem". In "implementation.py" muss die Klasse "DataSet" implementiert werden. "main.py" nutzt die Klassen "DataSet" und "DataSetItem" aus den jeweiligen Dateien und testet die Schnittstelle und Operationen von "DataSetInterface".

\subsection{Methoden}
Bei der Klasse "DataSet" sind insbesondere folgende Methoden zu implementieren (genaue Spezifikation in "dataset.py"):
\begin{itemize}
    \item \textbf{\_\_setitem\_\_(self, name, id\_content)} \\
    \begin{tabular}{p{\linewidth}}
    Hinzufügen eines Datums, mit Name, ID und Inhalt.
    \end{tabular}

    \item \textbf{\_\_iadd\_\_(self, item)} \\
    \begin{tabular}{p{\linewidth}}
    Hinzufügen eines "DataSetItem".
    \end{tabular}

    \item \textbf{\_\_delitem\_\_(self, name)} \\
    \begin{tabular}{p{\linewidth}}
    Löschen eines Datums.
    \end{tabular}

    \item \textbf{\_\_contains\_\_(self, name)} \\
    \begin{tabular}{p{\linewidth}}
    Prüfung, ob ein Datum vorhanden ist.
    \end{tabular}

    \item \textbf{\_\_getitem\_\_(self, name)} \\
    \begin{tabular}{p{\linewidth}}
    Abrufen des Datums.
    \end{tabular}

    \item \textbf{\_\_and\_\_(self, dataset)} \\
    \begin{tabular}{p{\linewidth}}
    Schnittmenge zweier Datensätze.
    \end{tabular}

    \item \textbf{\_\_or\_\_(self, dataset)} \\
    \begin{tabular}{p{\linewidth}}
    Vereinigung zweier Datensätze.
    \end{tabular}

    \item \textbf{\_\_iter\_\_(self)} \\
    \begin{tabular}{p{\linewidth}}
    Iteration über alle Daten.
    \end{tabular}

    \item \textbf{filtered\_iterate(self, filter)} \\
    \begin{tabular}{p{\linewidth}}
    Gefilterte Iteration.
    \end{tabular}

    \item \textbf{\_\_len\_\_(self)} \\
    \begin{tabular}{p{\linewidth}}
    Anzahl der Daten.
    \end{tabular}
\end{itemize}

\section{Abgabe}
Programmieren Sie die Klasse "DataSet" in der Datei "implementation.py" zur Lösung der oben beschriebenen Aufgabe. Sie können auch direkt auf Ihrem Computer programmieren; alle benötigten Dateien finden Sie im Moodle.
\par\vspace{\baselineskip} 
\noindent Das VPL nutzt den gleichen Code, wobei die "main.py" um weitere Testfälle und Überprüfungen erweitert wurde. Die Überprüfungen dienen dazu
sicherzustellen, dass Sie die richtigen Klassen nutzen.


\vspace{1cm}
\hrule
\vspace{1cm}

\textit{* Die Dateien befinden sich im Ordner "/code/" dieses Git-Repositories.}

\end{document}